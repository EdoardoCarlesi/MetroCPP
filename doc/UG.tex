\documentclass{article}
\usepackage{hyperref}
\usepackage[utf8]{inputenc}
\usepackage[english]{babel}
\usepackage[parfill]{parskip}

\pagenumbering{Roman}
\pagenumbering{arabic}
\title{\textbf{METROC++ v0.1} \\
\emph{A parallel code for the computation of merger trees in 
cosmological simulations}}
\author{Edoardo Carlesi\\ Leibniz Institut f\"ur Astrophysik\\ Potsdam, Germany\\ ecarlesi@aip.de}

\begin{document}
\maketitle
\tableofcontents
\newpage

\section{Introduction}
\texttt{METROC++} is an acronym that stands for \textbf{ME}rger\textbf{TR}ees \textbf{O}n \textbf{C++}, and refers to the 
infamous \textbf{Metro C} subway line in Rome, where the author of the code grew up and lived for more than 20 years.

This legendary piece of infrastructure (which is still - in 2019 - largely under construction, though a shorter section of 
it is already operating) took something of the order of a few Giga-Years to be built, a time span which can be compared to the
formation time of most dark matter halos. The name of the code is a tribute to this mythological pillar of the Roman public
transportation system, which connects through space different points of the city just like a merger tree connects a Halo 
through different points in time.\\

The code is distributed under the terms of the GNU Public License, and can be freely downloaded from 
this \texttt{GitHub} repository: \\

\noindent
\href{https://github.com/EdoardoCarlesi/MetroCPP}{https://github.com/EdoardoCarlesi/MetroCPP}.

\texttt{METROC++} has been written in \texttt{C++} and relies on \texttt{MPI2.0 C-}bindings for the parallelization; there are
also a number of \texttt{bash} scripts used to automatically and consistently read-in large number of halo and particle catalogs.

Since \texttt{C++11/14}-style syntax is used at some points, in particular for the manipulation
of (ordered) maps, the source code needs to be compiled with a non-prehistoric version of the compiler.
A series of python and \texttt{bash} scripts for the post processing, analysis and visualization of the merger trees are also provided 
together with the sources, inside the \texttt{scripts/} and \texttt{python/} subfolders. 
However, the post-processing scripts are very basic and might require a substantial tweak of the variables, so that the user
may prefer to write his own post-processing routines instead.\\

To address the issue of the so-called \emph{orphan} halos, i.e. halos (usually subhalos) unable to be linked to their parent 
halo for one or more time steps, the code keeps track of their particle content for a number of snapshots proportional to the
halo mass. In the usual mode of operation, though, it does not attempt to reconstruct the missing halos' positions and
velocity, a feature which is provided in the post processing mode of operation of the code.\\

A basic description of the algorithm and design of the code is provided in section \ref{sec:advanced}.
In this user's guide, we explain the basics for the setup and running of \texttt{METROC++}.


\section{Compiling and running the code}

The main folder contains a \texttt{Makefile.config} file with some options related to the performance of the code and the 
setup of the machine the code will be running on. Apart from a modern, $C++11$-compatible compiler and a working 
\texttt{MPI-2.0} installation, the code does not require the installation of any additional library.
The python scripts which are provided for post-processing might depend on libraries such as \texttt{numpy} and \texttt{matplotlib},
however these are still largely being coded and are provided only for a very quick-and-dirty peek into the output.

\noindent
\textbf{Compile-time options:} The \texttt{Makefile.config} file contains a few options that need to be switched on at runtime 
to optimize and speed-up the execution of the code

\noindent
\texttt{-DZOOM}: When this flag is switched on, the comparison algorithm is optimized for the 
zoom-in simulations, avoiding the buffer and overhead which is required by the full box comparison mode. 
Trees for zoom-in simulations can still be computed without switching this flag on, using one MPI task only.

\noindent
\texttt{-DVERBOSE}: This flag controls the output of the program - when enabled, the code will dump a lot more of 
(boring) information at runtime, on inter-task communication, buffer sizes, number of particles and halos exchanged and so on.

\noindent
\texttt{-DNPTYPES=N}: Here we control the number of particles being tracked separately. For reasons of internal consistency 
of the code, $N$ needs to be set greater or equal to $2$. Using the minimum amount of particle types required results in some marginal
gains in the memory usage and code speed.

\noindent
\texttt{-DNOPTYPE}: When this option is set, we disregard the type and do not differentiate between different particles, so that
each particle type has the same weight when computing the merger trees.

Once the adequate compile-time options are set, just type \texttt{make} in the root directory, and a binary executable named 
\texttt{MetroCPP} will be placed in the \texttt{bin/} subfolder. 



\textbf{Execution:}
To run the code, simply type:\\

\texttt{mpiexec -n $N_{MPI}$ ./bin/MetroCPP config/your\_configuration\_file.cfg}\\

\noindent
from the terminal in the installation directory of the code, using a properly setup for the configuration file.
The number $N_{MPI}$ of tasks on which the code should run on has to be chosen according to the number of \texttt{AHF} files each catalog
is split into. If each $AHF_halos$ file is divided into \texttt{nChunks} parts, the number $N_{MPI}$ 
of \texttt{MPI} tasks should be smaller than \texttt{nChunks} and allow for to be equally divided among the tasks.
For instance, if \texttt{nChunks}$=100$ setting $N_{MPI} = 10$ each MPI task would read in 10 pieces of the 
halo catalog at any given redshift.
Due to the fact that different tasks need to exchange buffers, an increase in the number of task does not necessarily mean
an increase in the speed of the code, as this will also increase the size of the buffer data that needs to be exchanged.



\subsection{Code operation modes}

The code has two basic modes of operation: \emph{tree-building} (\texttt{runMode=0}) and \emph{post-processing}
(\texttt{runMode=1}); \texttt{runMode=2} will execute the post-processing routine right after the tree-building is finished.
\textbf{The post-processing mode is not yet fully implemented in the code at the moment so that run modes 1 and 2 are disabled 
for the moment.} The post-processing routines will be smoothing out the mass accretion histories of the halos properly taking into 
account temporary fly-by of subhalos, bound satellites partially orbiting outside of the viral radius of their host (giving rise
to large mass fluctuations) and reconstruction of the orbits of the untracked halos.


\subsection{Configuration file}

The configuration file templates can be found in the subfolder \texttt{config/}. For the moment, these files are
designed for the \texttt{AHF} format only, but can be easily extended to other halo finders as well.
The expected structure of the input file is \texttt{prefix\_XXX.NNNN.zZ.ZZZ.suffix} (for those produced with the 
MPI version of \texttt{AHF}) or \texttt{prefix\_XXX.zZ.ZZZ.suffix} (for the serial AHF); where the \texttt{prefix} and
\texttt{suffix} are specified by the \texttt{haloPrefix}, \texttt{haloSuffix} for the halo catalogs and 
\texttt{partPrefix} and \texttt{partSuffix} for the particle lists.
Then we need to set the installation path of the program with the \texttt{pathMetroCpp} parameter, the path storing 
the input files (\texttt{pathInput}) and the path where the output files will be printed to, \texttt{pathOutput}.


\subsection{Temporary files}

The code produces a number of temporary files (\texttt{.tmp} format extension, located in the \texttt{tmp/} folder), containing
some information about the input files (such as their numbering and their redshift); these files are produced at each run but have
to be manually removed after each run. Otherwise the code will not attempt to write new files but rather read those already existing
inside the \texttt{tmp/} folder.
They can be produced manually or using the scripts (\texttt{find\_z.sh}, \texttt{find\_n.sh} located in the \texttt{scripts/} folder)
and contain the list of files on which the halo finder will run on.
This can be the full list of snapshots in one simulation, or can be edited to be only a subset of it.
These scripts produce three \texttt{.tmp} files in the \texttt{tmp/} subfolder called:
\texttt{output\_id.tmp} (which contains the snapshot unique number, usually in the form of \texttt{XXX} e.g. \texttt{001}, \texttt{023} 
\texttt{256} etc.); \texttt{output\_n.tmp} (that simply contains the total number of snapshots found, for consistency check)
\texttt{output\_z.tmp} (which stores the redshift corresponding to each snapshot).
These files can be also produced manually e.g. if the user wants to select a subset of snapshots for the comparison 
without using them at all the timesteps.
These files are read in at the beginning of each execution of the code; they are not needed for the subsequent run and 
should be deleted. They are not automatically erased each time in order to give to the users the opportunity to
edit or write their own list of catalogues to be compared.


\subsection{Input format}

\textbf{Halo catalogs}:
At the time being, \texttt{METROC++} supports \texttt{AHF} ASCII halo catalogs, of the form:
\begin{verbatim}
#ID(1)  hostHalo(2)     numSubStruct(3) Mvir(4) npart(5)        
Xc(6)   Yc(7)   Zc(8)   VXc(9)  VYc(10) VZc(11) Rvir(12) [...]
\end{verbatim} 
The reader can be easily modified for alternative halo catalog formats.


\textbf{Particle lists:}
Particle files must contain a list of the unique particle IDs that belong to each halo, and possibly their type 
(gas, dark matter, stars...). The first row of the file should contain the total number of halos stored in the
file; while the following lines should be structured as follows:

\begin{verbatim}
HaloID (1)  Particle Number (2)   
Particle ID (1) Particle Type (2)
\end{verbatim} 

\noindent
i.e. after specifying the halo ID and the number of particles it contains an actual list of particle IDs and types should follow.
When switching on the \texttt{-DNOPTYPE} option, only the particle ID will be read and the rest of the line will be ignored.



\subsection{Output format}
The code produces a series of \texttt{ASCII} files in the \texttt{.mtree} format 
as well as a logfile with the \texttt{.txt} extension.
The \texttt{.mtree} files contain basic information about the descendant-progenitor halos at each step, specified by the
header file printed by the master MPI task: \\

\begin{verbatim}
# ID host(1) N particles host(2) Num. progenitors(3)  
Orphan[0=no, 1=yes](4)
# Common DM particles (1) ID progenitor(2) Num. particles(3)
\end{verbatim} 

\noindent
In the case of orphan halos, which are signaled by the 1 on the fourth column, the information about the
dumped about the progenitor is a copy of the parent halo, keeping the same halo ID and number of particles.


\section{Examples}

\subsection{Full box simulation}
To properly run the code for full box simulations, the \texttt{-DZOOM} flag needs to be commented in the \texttt{Makefile.config} file.



\subsection{Zoom simulation}
To properly run the code on zoom simulation, the \texttt{-DZOOM} flag needs to be switched on in the \texttt{Makefile.config} file.
Zoom simulation can run on a single MPI task only, and the code needs to be executed as e.g.:\\

\texttt{mpiexec -n 1 ./bin/metroCPP config/zoom\_test.cfg}\\

This configuration of the code assumes that all the important halos are located in the same 

\section{Code properties}
\label{sec:advanced}


\subsection{Algorithms}
When dealing with cosmological full box simulations, the code reads in the halo catalogs among different tasks, so that
each task will only hold a fraction of the total halo and particle content of the simulation.
To ease the access and communication of data, a grid is placed on the simulation box on which each halo is assigned to its
nearest grid point (NGP). A node can be shared among different tasks.
Then, each task sends and receives the information for halos and particles at the edge of the subvolume of the box it contains. 
This buffer is needed to consistently compare the particle content of objects which may be close but located on different 
tasks at different steps. 


\subsection{Code structure}
The \texttt{src/} subfolder of the code contains the main source files:

\begin{itemize}
\item{\texttt{main.cpp}} A wrapper for the functions determined elsewhere
\item{\texttt{utils.cpp}} General functions and utilities are implemented here
\item{\texttt{spline.cpp}} The spline class is used for interpolation of e.g. $a(t)$ and $z(t)$, which are
read from tabulated quantities.
\item{\texttt{global\_vars.cpp}} A list of global variables accessible throughout the whole program
\item{\texttt{MergerTree.cpp}} This file contains the merger tree class (tracking the pairwise connections of 
halos in different catalogs) as well as functions (e.g. \texttt{FindProgenitors()}) 
that compute the  themselves.
\item{\texttt{IOSetting.cpp}} Input and output routines and settings.
\item{\texttt{Cosmology.cpp}} Everything related to cosmological calculations, gravity solver to reconstruct the orphan halo 
positions and velocities 
\item{\texttt{Communication.cpp}} Handles most of the communication among tasks - sending / receiving buffers and so on
\item{\texttt{Grid.cpp}} This class handles the grid on which halos are placed, and computes the buffer zones that need to 
be communicated among tasks. 
\item{\texttt{Halo.cpp}} The Halo class contains the basic halo properties and functions.
\end{itemize}

The \texttt{python/} subfolder contains some useful phyton libraries used to post process and store the data in 
a standard \texttt{SQL} database, to ease the access to the fully reconstructed merging histories querying the 
halo ID value of $z=0$ objects.


\subsection{Compatibility with other halo finders}
Although \texttt{METROC++} was conceived and mainly tested using the \texttt{AHF} halo finder, it can be easily extended to 
support other software as well, as long as:

\begin{itemize}
\item Halo catalogues include informations about the number and types of particles, positions and velocities for each object
\item Particle catalogs contain the (unique) IDs for each halo particles' content
\end{itemize}


\end{document}

